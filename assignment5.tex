\documentclass[journal,12pt,twocolumn]{IEEEtran}
\usepackage{amsthm}
\usepackage{gensymb}
\usepackage{setspace}
\singlespacing
\usepackage[cmex10]{amsmath}
\usepackage{bm}

\usepackage{cases}
\usepackage{mathrsfs}
\usepackage{cite}
\usepackage{stfloats}
\usepackage{mathtools}
\usepackage[breaklinks=true]{hyperref}
\usepackage{graphicx}
\usepackage{subfig}
\usepackage{txfonts}
\usepackage{longtable}
\usepackage{multirow}
\usepackage{tfrupee}
\usepackage{enumitem}
\usepackage{tikz}
\usepackage{steinmetz}
\usepackage{verbatim}
\usepackage{circuitikz}
\usepackage{tkz-euclide}





\usetikzlibrary{calc,math}
\usepackage{listings}
    \usepackage{color}                                            %%
    \usepackage{array}                                            %%
    \usepackage{longtable}                                        %%
    \usepackage{calc}                                             %%
    \usepackage{multirow}                                         %%
    \usepackage{hhline}                                           %%
    \usepackage{ifthen}                                           %%
    \usepackage{lscape}     
\usepackage{multicol}
\usepackage{chngcntr}

\DeclareMathOperator*{\Res}{Res}

\renewcommand\thesection{\arabic{section}}
\renewcommand\thesubsection{\thesection.\arabic{subsection}}
\renewcommand\thesubsubsection{\thesubsection.\arabic{subsubsection}}

\renewcommand \thesectiondis{\arabic{section}}
\renewcommand\thesubsectiondis{\thesectiondis.\arabic{subsection}}
\renewcommand\thesubsubsectiondis{\thesubsectiondis.\arabic{subsubsection}}


\hyphenation{op-tical net-works semi-conduc-tor}
\def\inputGnumericTable{}                                 %%

\lstset{
%language=C,
frame=single, 
breaklines=true,
columns=fullflexible
}
\begin{document}

\newcommand{\BEQA}{\begin{eqnarray}}
\newcommand{\EEQA}{\end{eqnarray}}
\newcommand{\define}{\stackrel{\triangle}{=}}
\bibliographystyle{IEEEtran}
\raggedbottom
\setlength{\parindent}{0pt}
\providecommand{\mbf}{\mathbf}
\providecommand{\pr}[1]{\ensuremath{\Pr\left(#1\right)}}
\providecommand{\qfunc}[1]{\ensuremath{Q\left(#1\right)}}
\providecommand{\sbrak}[1]{\ensuremath{{}\left[#1\right]}}
\providecommand{\lsbrak}[1]{\ensuremath{{}\left[#1\right.}}
\providecommand{\rsbrak}[1]{\ensuremath{{}\left.#1\right]}}
\providecommand{\brak}[1]{\ensuremath{\left(#1\right)}}
\providecommand{\lbrak}[1]{\ensuremath{\left(#1\right.}}
\providecommand{\rbrak}[1]{\ensuremath{\left.#1\right)}}
\providecommand{\cbrak}[1]{\ensuremath{\left\{#1\right\}}}
\providecommand{\lcbrak}[1]{\ensuremath{\left\{#1\right.}}
\providecommand{\rcbrak}[1]{\ensuremath{\left.#1\right\}}}
\theoremstyle{remark}
\newtheorem{rem}{Remark}
\newcommand{\sgn}{\mathop{\mathrm{sgn}}}
\providecommand{\abs}[1]{\vert#1\vert}
\providecommand{\res}[1]{\Res\displaylimits_{#1}} 
\providecommand{\norm}[1]{\lVert#1\rVert}
%\providecommand{\norm}[1]{\lVert#1\rVert}
\providecommand{\mtx}[1]{\mathbf{#1}}
\providecommand{\mean}[1]{E[ #1 ]}
\providecommand{\fourier}{\overset{\mathcal{F}}{ \rightleftharpoons}}
%\providecommand{\hilbert}{\overset{\mathcal{H}}{ \rightleftharpoons}}
\providecommand{\system}{\overset{\mathcal{H}}{ \longleftrightarrow}}
	%\newcommand{\solution}[2]{\textbf{Solution:}{#1}}
\newcommand{\solution}{\noindent \textbf{Solution: }}
\newcommand{\cosec}{\,\text{cosec}\,}
\providecommand{\dec}[2]{\ensuremath{\overset{#1}{\underset{#2}{\gtrless}}}}
\newcommand{\myvec}[1]{\ensuremath{\begin{pmatrix}#1\end{pmatrix}}}
\newcommand{\mydet}[1]{\ensuremath{\begin{vmatrix}#1\end{vmatrix}}}
\numberwithin{equation}{subsection}
\makeatletter
\@addtoreset{figure}{problem}
\makeatother
\let\StandardTheFigure\thefigure
\let\vec\mathbf
\renewcommand{\thefigure}{\theproblem}
\def\putbox#1#2#3{\makebox[0in][l]{\makebox[#1][l]{}\raisebox{\baselineskip}[0in][0in]{\raisebox{#2}[0in][0in]{#3}}}}
     \def\rightbox#1{\makebox[0in][r]{#1}}
     \def\centbox#1{\makebox[0in]{#1}}
     \def\topbox#1{\raisebox{-\baselineskip}[0in][0in]{#1}}
     \def\midbox#1{\raisebox{-0.5\baselineskip}[0in][0in]{#1}}
\vspace{3cm}
\title{Assignment5}%number
\author{CS20Btech11035 -NYALAPOGULA MANASWINI}
\maketitle
\newpage
\bigskip

\renewcommand{\thefigure}{\theenumi}
\renewcommand{\thetable}{\theenumi}
Download python code from 
\begin{lstlisting}
https://github.com/N-Manaswini23/assignment5/blob/main/assignment5.py
\end{lstlisting}
%
Download latex code from 
\begin{lstlisting}
https://github.com/N-Manaswini23/assignment5/blob/main/assignment5.tex
\end{lstlisting}
%

\section*{GATE 2021 ST QUESTION 44}
Let $(X,Y)$ have a bivariate normal distribution with the joint probability density function
\begin{align}
f_{X,Y}(x,y)=\frac{1}{\pi}e^{(\frac{3}{2}xy-\frac{25}{32}x^2-2y^2)}\\
-\infty < x,y < \infty
\end{align}
Then $E(XY)$ equals 
\section*{SOLUTION}
Given probability density function for $(X,Y)$
\begin{align}
f_{X,Y}(x,y)&=\frac{1}{\pi}e^{(\frac{3}{2}xy-\frac{25}{32}x^2-2y^2)} \label{1} \\ 
-\infty &< x,y < \infty
\end{align}
Joint pdf of bivariate normal distribution $N(\mu_x,\mu_y,\sigma_x^2,\sigma_y^2,\rho)$ is
\begin{align}
f_{X,Y}(x,y)&=\frac{1}{2\pi\sigma_x\sigma_y\sqrt{1-p^2}} \times \notag \\ &e^{\frac{-1}{2(1-p^2)}\sbrak{\sbrak{\frac{(x-\mu_x)}{\sigma_x}}^2+\sbrak{\frac{(y-\mu_y)}{\sigma_y}}^2-2\rho\sbrak{\frac{(x-\mu_x)}{\sigma_x}\frac{(y-\mu_y)}{\sigma_y}}}} \label{eq}
\end{align}
Comparing \eqref{eq} and \eqref{1} we get 

\begin{table}[h!]
\resizebox{7cm}{!}
{
\begin{tabular}{|c|c|c|c|c|}
\hline
$\mu_x$ & $\mu_y$ & $\sigma_x$ & $\sigma_y$ & $\rho$ \\
\hline
$0$ & $0$ & $1$ & $\frac{5}{8}$ & $\frac{3}{5}$\\
\hline
\end{tabular}
}
\caption{Table 1} 
\label{tab:1}
\end{table}

We need to find $E(XY)$
\begin{align}
E(XY)&= \int_{-\infty}^{+\infty}\int_{-\infty}^{+\infty} xyf_{X,Y}(x,y)\mathrm{d}y \mathrm{d}x \label{2}
\end{align}
Substiting \eqref{eq} in \eqref{2} and simplifing,we get
\begin{align}
E(XY)&=\rho \sigma_x\sigma_y+\mu_x\mu_y \label{eq:3}
\end{align}
Substituting table\eqref{tab:1} in \eqref{eq:3} we get
\begin{align}
E(XY)&=\frac{3}{8} \label{6}\\
\therefore 8E(XY)&=3
\end{align}



















\end{document}